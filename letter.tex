% Uncomment the following line to have comments hihglighted
\documentclass[9pt,draft]{extarticle}
% Uncomment the following line to disable comments and highlights
% \documentclass[9pt,final]{extarticle}

% To allow for extra UTF-8 characters in the text
\usepackage[utf8]{inputenc}
% To use colours
\usepackage[usenames,dvipsnames]{xcolor}
% To nicely format URLs
\usepackage{url}
% For using inparaenum, basically
\usepackage{paralist}
% For adding TODO notes
\usepackage[obeyDraft,textsize=tiny,backgroundcolor=blue!10]{todonotes}
% To 
% \usepackage[square,numbers,sectionbib]{natbib}
% To add author blocks to the front-matter
\usepackage{authblk}
% To create linked anchors on top of references
%\usepackage[pdfborder={0 0 0}]{hyperref}
% To play around with enumerations and bullet lists
\usepackage{enumitem}
% To load hyphenation rules and other Locale-standardised things
\usepackage[british]{babel}
% For the letter-like symbol, {\Letter}
\usepackage{marvosym}
% To comment
\usepackage{comment}
% To adjust margins
\usepackage{geometry}
% Fancy HREFs
\usepackage[draft=true]{hyperref}
%\usepackage{nohyperref}
% Fancy enumerations and itemised lists
\usepackage{enumitem}
% For \mathbb command, among the others
\usepackage{amsfonts}
% For creating side-notes
\usepackage{marginnote}
% For specifying kewords and acronyms
\usepackage[nonumberlist,acronym,sanitize=none]{glossaries}
\glsdisablehyper
% For equations, arrays of equations, defining operator names, etc.
\usepackage{amsmath}
% For cursive math
\usepackage{mathrsfs}
% For math symbols, such as \nexists
\usepackage{amssymb}
% To check whether the document is in its final version or not
\usepackage{ifdraft}
% To highlight text
\usepackage{soul}
% For a decent formatting of numbers (and a wonderful system for numeric columns in tables, ``S'')
\usepackage{siunitx}
% To create right arrows in the text environment
\usepackage{textcomp}
% For smart references
\usepackage[capitalise,nameinlink]{cleveref}
\crefname{algocf}{algorithm}{algorithms}
\Crefname{algocf}{Algorithm}{Algorithms}
% To cross-reference other documents
\usepackage{xr}

% requires packages:
% To check whether the document is in its final version or not
\RequirePackage{ifdraft}
% To color text
\RequirePackage{xcolor}
% To make smart references
\RequirePackage{cleveref}
%
\geometry{
 a4paper,
 total={210mm,297mm},
 left=15mm,
 right=15mm,
 top=15mm,
 bottom=25mm,
 marginparwidth=15mm
}
%
\renewcommand\Affilfont{\itshape\small}

% Counter for comments
\newcounter{commentcnt}
% Refine the comment counters by using the Roman numbering system. Notice that the new counter {commentcnt} automatically creates \thecommentcnt as a new command, which has to be overwritten.
\renewcommand{\thecommentcnt}{\Roman{commentcnt}}
% To use cleveref with this environment
\Crefname{commentcnt}{Comment}{Comment}

%%%%%%%%%%%%%%%%%%%%%%%%%%%%%%%%%%%%%%%%%%%%%%%%%%%%%%%%%%%%%%%%
%
% Environments for comments and replies
%
%%%%%%%%%%%%%%%%%%%%%%%%%%%%%%%%%%%%%%%%%%%%%%%%%%%%%%%%%%%%%%%%
\newenvironment{ReviewerComment}[2][]{\noindent\begin{minipage}[t]{\textwidth}\noindent \textbf{Comment \refstepcounter{commentcnt}{\thecommentcnt} %
% If an optional argument is passed, it is used as the label of the environment
\ifx\empty#1\relax\else#1\fi%
(Reviewer #2): } \begin{quotation}\noindent\begin{em}}{\end{em}\end{quotation}\vspace{1ex}\end{minipage}}
%
\newenvironment{ReviewerCommentReprise}{\noindent\vspace{-1.25cm}%
\begin{quotation}\noindent\begin{em}}{\end{em}\end{quotation}}
%
\newenvironment{ReviewerReply}{\noindent \textbf{Feedback: } \begin{quotation}\begin{em}}{\end{em}\end{quotation}}
%
\newenvironment{GuestEditorComment}{\noindent\begin{minipage}[t]{\textwidth}\noindent \textbf{Comment \refstepcounter{commentcnt}{\thecommentcnt} (Guest Editor): } \begin{quotation}\noindent\begin{em}}{\end{em}\end{quotation}\vspace{1ex}\end{minipage}}
%
\newenvironment{AssociateEditorComment}{\noindent\begin{minipage}[t]{\textwidth}\noindent \textbf{Comment \refstepcounter{commentcnt}{\thecommentcnt} (Associate Editor): } \begin{quotation}\noindent\begin{em}}{\end{em}\end{quotation}\vspace{1ex}\end{minipage}}
%
\newenvironment{AreaEditorComment}{\noindent\begin{minipage}[t]{\textwidth}\noindent \textbf{Comment \refstepcounter{commentcnt}{\thecommentcnt} (Area Editor): }\nopagebreak \begin{quotation}\noindent\begin{em}}{\end{em}\end{quotation}\vspace{1ex}\end{minipage}}
%
\newenvironment{EditorComment}{\noindent\begin{minipage}[t]{\textwidth}\noindent \textbf{Comment \refstepcounter{commentcnt}{\thecommentcnt} (Editor): } \begin{quotation}\noindent\begin{em}}{\end{em}\end{quotation}\vspace{1ex}\end{minipage}}
%
\newenvironment{EditorInChiefComment}{\noindent\begin{minipage}[t]{\textwidth}\noindent \textbf{Comment \refstepcounter{commentcnt}{\thecommentcnt} (Editor in Chief): } \begin{quotation}\noindent\begin{em}}{\end{em}\end{quotation}\vspace{1ex}\end{minipage}}
%
\newenvironment{MetaReviewComment}{\noindent\begin{minipage}[t]{\textwidth}\noindent \textbf{Comment \refstepcounter{commentcnt}{\thecommentcnt} (Meta-review): } \begin{quotation}\noindent\begin{em}}{\end{em}\end{quotation}\vspace{1ex}\end{minipage}}
%
\newenvironment{Answer}{\noindent \textbf{Answer: }}{\\[1cm]}
%
\newenvironment{AnswerInBetween}{\noindent \textbf{Answer: }}{\vspace{1cm}}
%
% \newcommand{\NoteInEvidence}[1]{\color{GreenYellow}{\textbf{#1}}}


\newcommand{\NoteInEvidence}[1]{\ifoptiondraft{\hl{#1}}{}}

%%%%%%%%%%%%%%%%%%%%%%%%%%%%%%%%%%%%%%%%%%%%%%%%%%%%%%%%%%%%%%%%
%
% Task difficulty assessment and work status boxes
%
%%%%%%%%%%%%%%%%%%%%%%%%%%%%%%%%%%%%%%%%%%%%%%%%%%%%%%%%%%%%%%%%
\newcommand{\TaskEstimationBox}[2]{%
\ifoptiondraft{\newline \parbox{1.0\linewidth}{\hfill \hfill {\colorbox{#2}{\color{White} \textbf{#1}}}}}%
{}%
}
%
\def\WorkInProgressRevTask {\TaskEstimationBox{Work in progress}{Cyan}}
\def\AlmostDoneRevTask {\TaskEstimationBox{Almost there}{NavyBlue}}
\def\RevTaskDone {\TaskEstimationBox{Done}{Blue}}
%
\def\NotEstimatedRevTask {\TaskEstimationBox{Effort not estimated}{Gray}}
\def\EasyRevTask {\TaskEstimationBox{Feasible}{ForestGreen}}
\def\MediumRevTask {\TaskEstimationBox{Medium effort}{Orange}}
\def\TimeConsumingRevTask {\TaskEstimationBox{Time-consuming}{Bittersweet}}
\def\HardRevTask {\TaskEstimationBox{Hard one}{Sepia}}
\def\DeathRevTask {\TaskEstimationBox{Death}{Black}}
%
\newcommand{\Assignment}[1]{
%
\ifoptiondraft{%
\newline \parbox{1.0\linewidth}{\hfill \hfill \textbf{Assignment:} #1}%
}{}%
}

%
%%%%%%%%%%%%%%%%%%%%%%%%%%%%%%%% Put some space to separate reviewers comments
%
\newcommand{\SkipSpaceForReviewerComments}{\vspace{6em}}

%
%%%%%%%%%%%%%%%%%%%%%%%%%%%%%%%% Add notes to be removed when the letter is not a draft any more
%
% To color text
\RequirePackage{xcolor}
\newenvironment{NoteForAuthors}%
  {\ifoptiondraft{%
      \noindent%
      \colorbox{gray}%
      {\color{white} Note: }%
      \color{orange}%
      \begin{em}%
    }{}%
  }%
  {\ifoptiondraft{%
      \normalcolor%
      \end{em}%
    }{}%
  }

%
%%%%%%%%%%%%%%%%%%%%%%%%%%%%%%%% Highlight changes in revised manuscripts
%
% To color text
\RequirePackage{xcolor}
% To create side-notes
\RequirePackage{marginnote}
% Change the following line for larger (or narrower) side spaces
\setlength{\marginparwidth}{1cm}
% Change the following line for bigger (or smaller) fonts
\renewcommand*{\marginfont}{\footnotesize}

\newenvironment{HlRev}[1][]{%
	% No hyperrefs here
	\begin{NoHyper}
		% Change the colour to olive green
		\color{OliveGreen}%
		% If an argument is passed, it is used as the number of the comment in a side note
		\ifx\newenvironment#1\newenvironment\else\marginnote{Comment #1}\fi%
		%
	}{%
		% Restores the normal colour
		\normalcolor%
		% Restores the normal hyperref behaviour
	\end{NoHyper}%
}
%
%%%%%%%%%%%%%%%%%%%%%%%%%%%%%%%%%%%%%%%%%%%%%%%%%%%%%%%%%%%%%%%%%%%%%%%%%
%

%
% Bibliography management
\usepackage[square,comma,numbers,sort,sectionbib,nonamebreak]{natbib}

% Defines the path for the main output file (so as to cross-reference sections, figures, table there, by prepending "paper:" to the original label )
\externaldocument[paper:]{../manuscript}

\def\PaperTitle{Title of the paper}
\def\PaperId{{JRNL\_YEAR\_NUM}}
\def\PaperRevision{1}
\def\SpecialIssueId{{SI:CONF20XX}}
\def\Journal{{Journal of Something}}

\def\AuthorsInLetter{Claudio {Di Ciccio}, Author Ecs, Writer Why, Contributor Zed}
\def\AuthorsInSignature{Claudio {Di Ciccio}, on behalf of the authors}


%opening
\title{\textbf{\PaperTitle} \\ {\Large Revision notes}}
%
%
\author[1]{Claudio~{Di~Ciccio}%
  \thanks{Corresponding author \\
    {\Letter}~{Utrecht University, Princetonplein 5, 3584 CC Utrecht, Netherlands} \\
    {\Email}~\texttt{\href{mailto:c.diciccio@uu.nl}{c.diciccio@uu.nl}} \\
    {\Telefon}~\textsf{+12 3 456 7890}%
  }
}
%
\author[2]{Author Ecs}
%
\author[3]{Writer Why}
%
\author[3]{Contributor Zed}
%
\affil[1]{Sapienza University of Rome, Italy}
\affil[2]{University of a City, World}
\affil[3]{The City University, World}

%
\date{}
\begin{document}

\maketitle

Dear Editors,%Dr.\ Motahari Nezhad,
\\[2em]
We express our gratitude for the time and effort dedicated to the reviewing of our submitted manuscript. We worked diligently to address all the concerns raised by the referees. Below we provide our detailed response to their comments.
%We have highlighted the main changes in the paper by colouring the modified text and adding a side note in which the addressed remark is referenced.
We hope that the applied revisions are to the satisfaction of the editors.
\\[2em]
Kind regards,
\begin{flushright}
\AuthorsInLetter
\end{flushright}
\vfill
\section*{Manuscript information}

\begin{description}
\item[Number:] \PaperId
\item[Title:] ``\PaperTitle''
\item[Authors:] \AuthorsInLetter
\item[Submitted to:] \Journal
\item[Special issue:] \SpecialIssueId
\end{description}
\vfill
\pagebreak

%\input{revision-notes/guesteditor}

%\SkipSpaceForReviewerComments

%
% !TeX root = ../letter.tex
% !TeX spellcheck = en_GB
%
%
%%%%%%%%%%%%%%%%%%%%%%%%%%%%%%%%%%%%%%%%%%%%%%%%%%%%%%%%%%%%%%%%%%%%%%%%%%%%%%%%%%%%%%%%%%%%%%%%%%%%%%%%%%%%%%%%%%%%%%%%%%%%%%%%
%
\begin{ReviewerComment}
This paper talks about something and is very good.\end{ReviewerComment}
%
\begin{Answer}
We thank the Reviewer for the thorough summary of the contents and the appreciation in our work.
\Assignment{Claudio} \RevTaskDone
\end{Answer}
%
%
%%%%%%%%%%%%%%%%%%%%%%%%%%%%%%%%%%%%%%%%%%%%%%%%%%%%%%%%%%%%%%%%%%%%%%%%%%%%%%%%%%%%%%%%%%%%%%%%%%%%%%%%%%%%%%%%%%%%%%%%%%%%%%%%
%


\SkipSpaceForReviewerComments

%
% !TeX root = ../letter.tex
% !TeX spellcheck = en_GB
%
\section*{Reviewer 2}
%
%%%%%%%%%%%%%%%%%%%%%%%%%%%%%%%%%%%%%%%%%%%%%%%%%%%%%%%%%%%%%%%%%%%%%%%%%%%%%%%%%%%%%%%%%%%%%%%%%%%%%%%%%%%%%%%%%%%%%%%%%%%%%%%%
%
\begin{ReviewerComment}{2}
The paper presents a breakthrough approach to tackle an interesting challenge. However, I have not understood what the challenge is.\end{ReviewerComment}
%
\begin{Answer}
We have worked on the paper in order to make the contribution clearer. We hope that the Reviewer is now satisfied by the modifications applied to this revision.
\Assignment{Claudio} \RevTaskDone
\end{Answer}
%
%%%%%%%%%%%%%%%%%%%%%%%%%%%%%%%%%%%%%%%%%%%%%%%%%%%%%%%%%%%%%%%%%%%%%%%%%%%%%%%%%%%%%%%%%%%%%%%%%%%%%%%%%%%%%%%%%%%%%%%%%%%%%%%%
%

\SkipSpaceForReviewerComments

%
% !TeX root = ../letter.tex
% !TeX spellcheck = en_GB
%
%
%%%%%%%%%%%%%%%%%%%%%%%%%%%%%%%%%%%%%%%%%%%%%%%%%%%%%%%%%%%%%%%%%%%%%%%%%%%%%%%%%%%%%%%%%%%%%%%%%%%%%%%%%%%%%%%%%%%%%%%%%%%%%%%%
\begin{ReviewerComment}{3}
The manuscript has a very intriguing title.\end{ReviewerComment}
%
\begin{Answer}
We express our gratitude to the Reviewer for the appreciation in our work.
\Assignment{Claudio} \RevTaskDone
\end{Answer}
%
%
%%%%%%%%%%%%%%%%%%%%%%%%%%%%%%%%%%%%%%%%%%%%%%%%%%%%%%%%%%%%%%%%
% Back matter
%%%%%%%%%%%%%%%%%%%%%%%%%%%%%%%%%%%%%%%%%%%%%%%%%%%%%%%%%%%%%%%%
%
%
% Remove it when you are done
\ifoptiondraft{
	\pagebreak
	\section*{How to use commands for this response letter}
	% !TeX root = letter.tex
%%%%%%%%%%%%%%%%%%%%%%%%%%%%%%%%%%%%%%%%%%%%%%%%%%%%%%%%%%%%%%%%%%%%%%%%%%%%%%%%%%%%%%%%%%%%%%%%%%%%%%%%%%%%%%%%%%%%%%%%%%%%%%%%
%
\begin{NoteForAuthors}
	If you wish to remove this instructions page, just turn the \verb|\documentclass| command in \texttt{letter.tex}
	\\
	from \verb|\documentclass[9pt,draft]{extarticle}|
	\\
	to \verb|\documentclass[9pt,final]{extarticle}|
\end{NoteForAuthors}
\bigskip

\noindent
Begin a new section for another reviewer with \verb*|\ReviewerSection|. This command will print a new unnumbered section's title and advance a special counter named \verb*|thereviewer|. See below.

\ReviewerSection
\begin{ReviewerComment}[\label{memo:example:comment}]
Comment of Reviewer \Roman{thereviewer}.
\end{ReviewerComment}
%
\begin{Answer}
\begin{sloppypar}
Reply of the authors. Notice that the Reviewer's number is given by the counter (\verb*|thereviewer|). The comment's counter increases automatically at every new \verb*|\begin{ReviewerComment}...\end{ReviewerComment}|. In the {\LaTeX} code of this very part, you might notice that we pass an optional parameter when beginning the \verb*|ReviewerComment| environment: \verb*|\label{memo:example:comment}|. It is a label to be used later on for references (\verb*|ref| and \verb*|cref|; see below).
%
%
\paragraph{Task assignment and degree of difficulty.}
Below it, please assign the task to one of the authors, and define one of the following (expected) levels of difficulty.
For instance:
\verb|\Assignment{?} \NotEstimatedRevTask|
for unassigned/not assessed revision tasks, or
\verb|\Assignment{Claudio} \MediumRevTask|
otherwise.
When a revision task is done, you can mark it with ``Done'': \verb|\RevTaskDone|.
When a revision task is work-in-progress, you can mark it with ``Work-in-progress'': \verb|\WorkInProgressRevTask|.
To mark a revision task as almost done, you can use the command \verb|\AlmostDoneRevTask|.
All boxes about the difficulty or the status of the revision task will disappear by adding the \verb|final| option to the document.
%
%
\paragraph{Highlighted notes.}
If there are some notes that should be highlighted, say, in the answer to reviewers, please use the commands defined on purpose,
\\
\NoteInEvidence{like this:}
\verb|\NoteInEvidence{like this:}|
\\
\begin{NoteForAuthors}or like this.\end{NoteForAuthors}
\verb|\begin{NoteForAuthors}or like this.\end{NoteForAuthors}|
\\
Both kinds of highlights will be removed as soon as the ``draft'' option is removed from the document class preamble declaration.
\\
\begin{HlRev}[\ref{memo:example:comment}]
Should you want to emphasise some changes in the manuscript for further check, discussing with the other co-authors, etc., please enclose the text to be highlighted in \ldots
\end{HlRev}
\begin{verbatim}
\begin{HlRev}[\ref{memo:example:comment}]
... something like this. The parameter (here \ref{memo:example:comment}) is optional. It is put after
"Comment" as a side-note. Feel free to pass multiple comma-separated ref's! \end{HlRev}
\end{verbatim}
%
Do you like it? To make it work, make sure to import the \verb|addon/commands-for-revisions.tex| file in your manuscript's main file's preamble, too, and alter the \verb|\externaldocument| command indicating the location of the response letter's file. 
%
\paragraph{References.}
Please notice that to refer to comments within the document, they first must be labelled with an optional parameter:
\begin{verbatim}
\begin{ReviewerComment}[\label{memo:example:comment}]
\end{verbatim}
Later on, also the \verb|\cref{memo:example:comment}| command can be used, to automatically obtain: \cref{memo:example:comment}.

To mention sections, figures, etc., please use the \verb|\ref| command as usual: When referring to the manuscript, prepend \verb|paper:| as a prefix before the actual section/figure/etc. label ID. For instance, the ``\ref{paper:sec:introduction}'' of ``Section~\ref{paper:sec:introduction}''%
\footnote{If you read ``??'' it must be because the referenced paper has no section labelled as ``sec:introduction''.
}
is here generated by this {\LaTeX} code:
\verb|\ref{paper:sec:introduction}|
%
Please alter the line in \texttt{letter.tex} according to the location of the main paper file, so as to let the reference be found:
\begin{verbatim}
	\externaldocument[paper:]{../manuscript}
\end{verbatim}
Bibliographic references are, of course, welcome (e.g., we can mention here a few articles and conference papers whose revisions were accompanied by letters created with this template~\cite{Yeshchenko.etal/TVCG2022:VisualDriftDetectionEventSequenceDataBusinessProcesses,Cecconi.etal/IS2024:MeasuringRuleBasedLTLfSpecifications,Basile.DiCiccio/BPM2025:SecrecyPreservationOnline,}). See the lines at the end of the \textit{letter.tex} (main) file.
%
\paragraph{Other ``comment'' environments.}
Notice that there are several environments other than \verb|ReviewerComment| to specify the role of who wrote the review, such as \verb|GuestEditorComment|, \verb|MetaReviewComment|, \verb|EditorComment|, and the others you can find in: \texttt{addon/commands-for-letter-to-reviewers.tex}. It is advisable not to introduce those with \verb*|\ReviewerSection| unless you want the reviewer's counter to increase accordingly. A regular \verb*|\section*{...}| (e.g., \verb*|\section*{Meta-review}| or \verb|\section*{Editor's comments}|) will do. Comment headers will then be printed without any preceding roman digits (e.g., numbered with \arabic{commentcnt} in place of \Roman{thereviewer}.\arabic{commentcnt}).
\end{sloppypar}
In the following, a couple examples are given. Later on in this page, the different levels of difficulty and the statuses of the referred modifications are listed. Look at the {\LaTeX} code of this page for more information.
\end{Answer}
%
%%%%%%%%%%%%%%%%%%%%%%%%%%%%%%%%%%%%%%%%%%%%%%%%%%%%%%%%%%%%%%%%%%%%%%%%%%%%%%%%%%%%%%%%%%%%%%%%%%%%%%%%%%%%%%%%%%%%%%%%%%%%%%%%
%
\begin{ReviewerComment}
	Another comment of Reviewer \Roman{thereviewer}.
\end{ReviewerComment}
%
\begin{Answer}
	Reply to Reviewer \Roman{thereviewer}.
	\Assignment{Author} \NotEstimatedRevTask
\end{Answer}
%
%%%%%%%%%%%%%%%%%%%%%%%%%%%%%%%%%%%%%%%%%%%%%%%%%%%%%%%%%%%%%%%%%%%%%%%%%%%%%%%%%%%%%%%%%%%%%%%%%%%%%%%%%%%%%%%%%%%%%%%%%%%%%%%%
%
\begin{ReviewerComment}
Start of the reviewer's comment.
\end{ReviewerComment}
%
\begin{AnswerInBetween}
In-between reply of the authors.
\Assignment{Author} \HardRevTask
\end{AnswerInBetween}
%
%%%%%%%%%%%%%%%%%%%%%%%%%%%%%%%%%%%%%%%%%%%%%%%%%%%%%%%%%%%%%%%%
%
\begin{ReviewerCommentReprise}
Reprise of the reviewer's comment.
\end{ReviewerCommentReprise}
%
\begin{AnswerInBetween}
In-between (or final) reply of the authors.
\Assignment{Author} \RevTaskDone
\end{AnswerInBetween}
\Assignment{Author}
\\
\NotEstimatedRevTask \EasyRevTask \MediumRevTask \TimeConsumingRevTask \HardRevTask \DeathRevTask 
\\
\WorkInProgressRevTask \AlmostDoneRevTask \RevTaskDone

	\pagebreak
}{}
%
%%%%%%%%%%%%%%%%%%%%%%%%%%%%%%%%
% Bibliography
%%%%%%%%%%%%%%%%%%%%%%%%%%%%%%%%
%
\bibliographystyle{unsrtnat}
\bibliography{bibliography/bib}

\end{document}
