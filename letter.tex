% Uncomment the following line to have comments hihglighted
\documentclass[9pt,draft]{extarticle}
% Uncomment the following line to disable comments and highlights
% \documentclass[9pt,final]{extarticle}

\input{addon/pkgloading-for-letter-to-reviewers}
% requires packages:
% To check whether the document is in its final version or not
\RequirePackage{ifdraft}
% To color text
\RequirePackage{xcolor}
% To make smart references
\RequirePackage{cleveref}
%
\geometry{
 a4paper,
 total={210mm,297mm},
 left=15mm,
 right=15mm,
 top=15mm,
 bottom=25mm,
 marginparwidth=15mm
}
%
\renewcommand\Affilfont{\itshape\small}

% Counter for comments
\newcounter{commentcnt}
% Refine the comment counters by using the Roman numbering system. Notice that the new counter {commentcnt} automatically creates \thecommentcnt as a new command, which has to be overwritten.
\renewcommand{\thecommentcnt}{\Roman{commentcnt}}
% To use cleveref with this environment
\Crefname{commentcnt}{Comment}{Comments}

%%%%%%%%%%%%%%%%%%%%%%%%%%%%%%%%%%%%%%%%%%%%%%%%%%%%%%%%%%%%%%%%
%
% Environments for comments and replies
%
%%%%%%%%%%%%%%%%%%%%%%%%%%%%%%%%%%%%%%%%%%%%%%%%%%%%%%%%%%%%%%%%
\newenvironment{ReviewerComment}[2][]{\noindent\begin{minipage}[t]{\textwidth}\noindent \textbf{Comment \refstepcounter{commentcnt}{\thecommentcnt}%
% If an optional argument is passed, it is used as the label of the environment
\ifx\empty#1\relax\else#1\fi% 
:} \begin{quotation}\noindent\begin{em}}{\end{em}\end{quotation}\vspace{1ex}\end{minipage}}
%
\newenvironment{ReviewerCommentReprise}{\noindent\vspace{-1.25cm}%
\begin{quotation}\noindent\begin{em}}{\end{em}\end{quotation}}
%
\newenvironment{ReviewerReply}{\noindent \textbf{Feedback: } \begin{quotation}\begin{em}}{\end{em}\end{quotation}}
%
\newenvironment{GuestEditorComment}{\noindent\begin{minipage}[t]{\textwidth}\noindent \textbf{Comment \refstepcounter{commentcnt}{\thecommentcnt} (Guest Editor): } \begin{quotation}\noindent\begin{em}}{\end{em}\end{quotation}\vspace{1ex}\end{minipage}}
%
\newenvironment{AssociateEditorComment}{\noindent\begin{minipage}[t]{\textwidth}\noindent \textbf{Comment \refstepcounter{commentcnt}{\thecommentcnt} (Associate Editor): } \begin{quotation}\noindent\begin{em}}{\end{em}\end{quotation}\vspace{1ex}\end{minipage}}
%
\newenvironment{AreaEditorComment}{\noindent\begin{minipage}[t]{\textwidth}\noindent \textbf{Comment \refstepcounter{commentcnt}{\thecommentcnt} (Area Editor): }\nopagebreak \begin{quotation}\noindent\begin{em}}{\end{em}\end{quotation}\vspace{1ex}\end{minipage}}
%
\newenvironment{EditorComment}{\noindent\begin{minipage}[t]{\textwidth}\noindent \textbf{Comment \refstepcounter{commentcnt}{\thecommentcnt} (Editor): } \begin{quotation}\noindent\begin{em}}{\end{em}\end{quotation}\vspace{1ex}\end{minipage}}
%
\newenvironment{EditorInChiefComment}{\noindent\begin{minipage}[t]{\textwidth}\noindent \textbf{Comment \refstepcounter{commentcnt}{\thecommentcnt} (Editor in Chief): } \begin{quotation}\noindent\begin{em}}{\end{em}\end{quotation}\vspace{1ex}\end{minipage}}
%
\newenvironment{MetaReviewComment}{\noindent\begin{minipage}[t]{\textwidth}\noindent \textbf{Comment \refstepcounter{commentcnt}{\thecommentcnt} (Meta-review): } \begin{quotation}\noindent\begin{em}}{\end{em}\end{quotation}\vspace{1ex}\end{minipage}}
%
\newenvironment{Answer}{\noindent \textbf{Answer: }}{\\[1cm]}
%
\newenvironment{AnswerInBetween}{\noindent \textbf{Answer: }}{\vspace{1cm}}
%
% \newcommand{\NoteInEvidence}[1]{\color{GreenYellow}{\textbf{#1}}}


\newcommand{\NoteInEvidence}[1]{\ifoptiondraft{\hl{#1}}{}}

%%%%%%%%%%%%%%%%%%%%%%%%%%%%%%%%%%%%%%%%%%%%%%%%%%%%%%%%%%%%%%%%
%
% Task difficulty assessment and work status boxes
%
%%%%%%%%%%%%%%%%%%%%%%%%%%%%%%%%%%%%%%%%%%%%%%%%%%%%%%%%%%%%%%%%
\newcommand{\TaskEstimationBox}[2]{%
\ifoptiondraft{\newline \parbox{1.0\linewidth}{\hfill \hfill {\colorbox{#2}{\color{White} \textbf{#1}}}}}%
{}%
}
%
\def\WorkInProgressRevTask {\TaskEstimationBox{Work in progress}{Cyan}}
\def\AlmostDoneRevTask {\TaskEstimationBox{Almost there}{NavyBlue}}
\def\RevTaskDone {\TaskEstimationBox{Done}{Blue}}
%
\def\NotEstimatedRevTask {\TaskEstimationBox{Effort not estimated}{Gray}}
\def\EasyRevTask {\TaskEstimationBox{Feasible}{ForestGreen}}
\def\MediumRevTask {\TaskEstimationBox{Medium effort}{Orange}}
\def\TimeConsumingRevTask {\TaskEstimationBox{Time-consuming}{Bittersweet}}
\def\HardRevTask {\TaskEstimationBox{Hard one}{Sepia}}
\def\DeathRevTask {\TaskEstimationBox{Death}{Black}}
%
\newcommand{\Assignment}[1]{
%
\ifoptiondraft{%
\newline \parbox{1.0\linewidth}{\hfill \hfill \textbf{Assignment:} #1}%
}{}%
}

%
%%%%%%%%%%%%%%%%%%%%%%%%%%%%%%%% Put some space to separate reviewers comments
%
\newcommand{\SkipSpaceForReviewerComments}{\vspace{6em}}

%
%%%%%%%%%%%%%%%%%%%%%%%%%%%%%%%% Add notes to be removed when the letter is not a draft any more
%
% To color text
\RequirePackage{xcolor}
\newenvironment{NoteForAuthors}%
  {\ifoptiondraft{%
      \noindent%
      \colorbox{gray}%
      {\color{white} Note: }%
      \color{orange}%
      \begin{em}%
    }{}%
  }%
  {\ifoptiondraft{%
      \normalcolor%
      \end{em}%
    }{}%
  }

%
%%%%%%%%%%%%%%%%%%%%%%%%%%%%%%%% Highlight changes in revised manuscripts
%
% To color text
\RequirePackage{xcolor}
% To create side-notes
\RequirePackage{marginnote}

%%%% Change the following line for larger (or narrower) side spaces
\setlength{\marginparwidth}{1cm}
\def\HlRevInnerWidth{0.75cm}
% Change the following line for bigger (or smaller) fonts
\renewcommand*{\marginfont}{\footnotesize}

\definecolor{burntorange}{rgb}{0.8, 0.33, 0.0}
\ExplSyntaxOn
\newenvironment{HlRev}[1][]{%
		% Change the colour to orange
		\color{burntorange}%
		% If an argument is passed, it is used as the number of the comment in a side note
		\ifx\newenvironment#1\newenvironment\else\marginnote{\parbox[t]{\HlRevInnerWidth}{% \raggedright%
			% No hyperrefs in side-notes
			\begin{NoHyper}
				\clist_set:Nn \l_tmpa_clist {#1}% Get the comma-separated list of comment references
				\int_set:Nn \l_tmpa_int { \clist_count:N \l_tmpa_clist }% Count the number of items in the list
				\hspace*{0pt}% Trick to allow the first word to be hyphenised if need be
				Comment% Write ``Comment''
				\int_compare:nT { \l_tmpa_int > 1 }{s}% Add an s (Comment*s*) if there are multiple comments inside
				{\ #1}% Write the referred comment(s)
			% Restores the normal hyperref behaviour
			\end{NoHyper}%
		}}\fi%
		%
	}{%
		% Restores the normal colour
		\normalcolor%
}%
\ExplSyntaxOff
%
%%%%%%%%%%%%%%%%%%%%%%%%%%%%%%%%%%%%%%%%%%%%%%%%%%%%%%%%%%%%%%%%%%%%%%%%%
%

%
% Bibliography management
\usepackage[square,comma,numbers,sort,sectionbib,nonamebreak]{natbib}

% Defines the path for the main output file (so as to cross-reference sections, figures, table there, by prepending "paper:" to the original label )
\externaldocument[paper:]{../manuscript}

\def\PaperTitle{Title of the paper}
\def\PaperId{{JRNL\_YEAR\_NUM}}
\def\PaperRevision{1}
\def\SpecialIssueId{{SI:CONF20XX}}
\def\Journal{{Journal of Something}}

\def\AuthorsInLetter{Claudio {Di Ciccio}, Author Ecs, Writer Why, Contributor Zed}
\def\AuthorsInSignature{Claudio {Di Ciccio}, on behalf of the authors}


%opening
\title{\textbf{\PaperTitle} \\ {\Large Revision notes}}
%
%
\author[1]{Claudio~{Di~Ciccio}%
  \thanks{Corresponding author \\
    {\Letter}~{Utrecht University, Princetonplein 5, 3584 CC Utrecht, Netherlands} \\
    {\Email}~\texttt{\href{mailto:c.diciccio@uu.nl}{c.diciccio@uu.nl}} \\
    {\Telefon}~\textsf{+12 3 456 7890}%
  }
}
%
\author[2]{Author Ecs}
%
\author[3]{Writer Why}
%
\author[3]{Contributor Zed}
%
\affil[1]{Sapienza University of Rome, Italy}
\affil[2]{University of a City, World}
\affil[3]{The City University, World}

%
\date{}
\begin{document}

\maketitle

\noindent
Dear Editors,
\\[2em]
We express our gratitude for the time and effort dedicated to the reviewing of our submitted manuscript. We worked diligently to address all the concerns raised by the referees. Below we provide our detailed response to their comments.
We have highlighted the main changes in the paper by colouring the modified text and adding side notes in which the addressed remarks are referenced.
We hope that the applied revisions are to the satisfaction of the editors.
\\[2em]
With kind regards,
\begin{flushright}
	\AuthorsInSignature
\end{flushright}

\vfill
\input{manuscriptinfo}
\vfill
\pagebreak

%\input{revision-notes/guesteditor}

%\SkipSpaceForReviewerComments

\input{revision-notes/reviewer1}

\SkipSpaceForReviewerComments

\input{revision-notes/reviewer2}

\SkipSpaceForReviewerComments

\input{revision-notes/reviewer3}
%
%
%%%%%%%%%%%%%%%%%%%%%%%%%%%%%%%%%%%%%%%%%%%%%%%%%%%%%%%%%%%%%%%%
% Back matter
%%%%%%%%%%%%%%%%%%%%%%%%%%%%%%%%%%%%%%%%%%%%%%%%%%%%%%%%%%%%%%%%
%
%
% Remove it when you are done
\ifoptiondraft{
	\pagebreak
	\section*{How to use commands for this response letter}
	% !TeX root = letter.tex
%%%%%%%%%%%%%%%%%%%%%%%%%%%%%%%%%%%%%%%%%%%%%%%%%%%%%%%%%%%%%%%%%%%%%%%%%%%%%%%%%%%%%%%%%%%%%%%%%%%%%%%%%%%%%%%%%%%%%%%%%%%%%%%%
%
This template was used, among others, for the letters produced during the revision rounds of~\cite{Cecconi.etal/IS2024:MeasuringRuleBasedLTLfSpecifications}.
\begin{NoteForAuthors}
	If you wish to remove this instructions page, just turn the \verb|\documentclass| command in \texttt{letter.tex}
	\\
	from \verb|\documentclass[9pt,draft]{extarticle}|
	\\
	to \verb|\documentclass[9pt,final]{extarticle}|
\end{NoteForAuthors}
\bigskip

\begin{ReviewerComment}[\label{memo:example:comment}]{N}
Comment of Reviewer N.
\end{ReviewerComment}
%
\begin{Answer}
\begin{sloppypar}
Reply of the authors.
%
%
\paragraph{Task assignment and degree of difficulty.}
Below it, please assign the task to one of the authors, and define one of the following (expected) levels of difficulty.
For instance:
\verb|\Assignment{?} \NotEstimatedRevTask|
for unassigned/not assessed revision tasks, or
\verb|\Assignment{Claudio} \MediumRevTask|
otherwise.
When a revision task is done, you can mark it with ``Done'': \verb|\RevTaskDone|.
When a revision task is work-in-progress, you can mark it with ``Work-in-progress'': \verb|\WorkInProgressRevTask|.
To mark a revision task as almost done, you can use the command \verb|\AlmostDoneRevTask|.
All boxes about the difficulty or the status of the revision task will disappear by adding the \verb|final| option to the document.
%
%
\paragraph{Highlighted notes.}
If there are some notes that should be highlighted, say, in the answer to reviewers, please use the commands defined on purpose,
\\
\NoteInEvidence{like this:}
\verb|\NoteInEvidence{like this:}|
\\
\begin{NoteForAuthors}or like this.\end{NoteForAuthors}
\verb|\begin{NoteForAuthors}or like this.\end{NoteForAuthors}|
\\
Both kinds of highlights will be removed as soon as the ``draft'' option is removed from the document class preamble declaration.
\\
\begin{HlRev}[\ref{memo:example:comment}]
Should you want to emphasise some changes in the manuscript for further check, discussing with the other co-authors, etc., please enclose the text to be highlighted in \ldots
\end{HlRev}
\begin{verbatim}
\begin{HlRev}[\ref{memo:example:comment}]
... something like this. The parameter (here \ref{memo:example:comment}) is optional. It is put after
"Comment" as a side-note. Feel free to pass multiple comma-separated ref's! \end{HlRev}
\end{verbatim}
%
Do you like it? To make it work, make sure to import the \verb|addon/commands-for-revisions.tex| file in your manuscript's main file's preamble, too, and alter the \verb|\externaldocument| command indicating the location of the response letter's file. 
%
\paragraph{References.}
Please notice that to refer to comments within the document, they first must be labelled with an optional parameter:
\begin{verbatim}
\begin{ReviewerComment}[\label{memo:example:comment}]{N}
\end{verbatim}
Later on, also the \verb|\cref{memo:example:comment}| command can be used, to automatically obtain: \cref{memo:example:comment}.

To mention sections, figures, etc., please use the \verb|\ref| command as usual: When referring to the manuscript, prepend \verb|paper:| as a prefix before the actual section/figure/etc. label ID. For instance, the ``\ref{paper:sec:introduction}'' of ``Section~\ref{paper:sec:introduction}''%
\footnote{If you read ``??'' it must be because the referenced paper has no section labelled as ``sec:introduction''.
}
is here generated by this {\LaTeX} code:
\verb|\ref{paper:sec:introduction}|
%
Please alter the line in \texttt{letter.tex} according to the location of the main paper file, so as to let the reference be found:
\begin{verbatim}
	\externaldocument[paper:]{../manuscript}
\end{verbatim}
%
\paragraph{Other ``comment'' environments.}
Notice that there are several environments other than \verb|ReviewerComment| to specify the role of who wrote the review, such as \verb|GuestEditorComment|, \verb|MetaReviewComment|, \verb|EditorComment|, and the others you can find in:\\
\texttt{addon/commands-for-letter-to-reviewers.tex}.
\end{sloppypar}
In the following, a couple examples are given. Later on in this page, the different levels of difficulty and the statuses of the referred modifications are listed.
\end{Answer}
%
%%%%%%%%%%%%%%%%%%%%%%%%%%%%%%%%%%%%%%%%%%%%%%%%%%%%%%%%%%%%%%%%%%%%%%%%%%%%%%%%%%%%%%%%%%%%%%%%%%%%%%%%%%%%%%%%%%%%%%%%%%%%%%%%
%
\begin{ReviewerComment}{N}
	Comment of Reviewer N.
\end{ReviewerComment}
%
\begin{Answer}
	Reply to Reviewer.
	\Assignment{Author} \NotEstimatedRevTask
\end{Answer}
%
%%%%%%%%%%%%%%%%%%%%%%%%%%%%%%%%%%%%%%%%%%%%%%%%%%%%%%%%%%%%%%%%%%%%%%%%%%%%%%%%%%%%%%%%%%%%%%%%%%%%%%%%%%%%%%%%%%%%%%%%%%%%%%%%
%
\begin{ReviewerComment}{N}
Start of the reviewer's comment.
\end{ReviewerComment}
%
\begin{AnswerInBetween}
In-between reply of the authors.
\Assignment{Author} \HardRevTask
\end{AnswerInBetween}
%
%%%%%%%%%%%%%%%%%%%%%%%%%%%%%%%%%%%%%%%%%%%%%%%%%%%%%%%%%%%%%%%%
%
\begin{ReviewerCommentReprise}
Reprise of the reviewer's comment.
\end{ReviewerCommentReprise}
%
\begin{AnswerInBetween}
In-between (or final) reply of the authors.
\Assignment{Author} \RevTaskDone
\end{AnswerInBetween}
\Assignment{Author}
\\
\NotEstimatedRevTask \EasyRevTask \MediumRevTask \TimeConsumingRevTask \HardRevTask \DeathRevTask 
\\
\WorkInProgressRevTask \AlmostDoneRevTask \RevTaskDone

	\pagebreak
}{}
%
%%%%%%%%%%%%%%%%%%%%%%%%%%%%%%%%
% Bibliography
%%%%%%%%%%%%%%%%%%%%%%%%%%%%%%%%
%
\bibliographystyle{unsrtnat}
\bibliography{bibliography/bib}

\end{document}
