% !TeX root = letter.tex
%%%%%%%%%%%%%%%%%%%%%%%%%%%%%%%%%%%%%%%%%%%%%%%%%%%%%%%%%%%%%%%%%%%%%%%%%%%%%%%%%%%%%%%%%%%%%%%%%%%%%%%%%%%%%%%%%%%%%%%%%%%%%%%%
%
\begin{NoteForAuthors}
	If you wish to remove this instructions page, just turn the \verb|\documentclass| command in \texttt{letter.tex}
	\\
	from \verb|\documentclass[9pt,draft]{extarticle}|
	\\
	to \verb|\documentclass[9pt,final]{extarticle}|
\end{NoteForAuthors}
\bigskip

\noindent
Begin a new section for another reviewer with \verb*|\ReviewerSection|. This command will print a new unnumbered section's title and advance a special counter named \verb*|thereviewer|. See below.

\ReviewerSection
\begin{ReviewerComment}[\label{memo:example:comment}]
Comment of Reviewer \Roman{thereviewer}.
\end{ReviewerComment}
%
\begin{Answer}
\begin{sloppypar}
Reply of the authors. Notice that the Reviewer's number is given by the counter (\verb*|thereviewer|). The comment's counter increases automatically at every new \verb*|\begin{ReviewerComment}...\end{ReviewerComment}|. In the {\LaTeX} code of this very part, you might notice that we pass an optional parameter when beginning the \verb*|ReviewerComment| environment: \verb*|\label{memo:example:comment}|. It is a label to be used later on for references (\verb*|ref| and \verb*|cref|; see below).
%
%
\paragraph{Task assignment and degree of difficulty.}
Below it, please assign the task to one of the authors, and define one of the following (expected) levels of difficulty.
For instance:
\verb|\Assignment{?} \NotEstimatedRevTask|
for unassigned/not assessed revision tasks, or
\verb|\Assignment{Claudio} \MediumRevTask|
otherwise.
When a revision task is done, you can mark it with ``Done'': \verb|\RevTaskDone|.
When a revision task is work-in-progress, you can mark it with ``Work-in-progress'': \verb|\WorkInProgressRevTask|.
To mark a revision task as almost done, you can use the command \verb|\AlmostDoneRevTask|.
All boxes about the difficulty or the status of the revision task will disappear by adding the \verb|final| option to the document.
%
%
\paragraph{Highlighted notes.}
If there are some notes that should be highlighted, say, in the answer to reviewers, please use the commands defined on purpose,
\\
\NoteInEvidence{like this:}
\verb|\NoteInEvidence{like this:}|
\\
\begin{NoteForAuthors}or like this.\end{NoteForAuthors}
\verb|\begin{NoteForAuthors}or like this.\end{NoteForAuthors}|
\\
Both kinds of highlights will be removed as soon as the ``draft'' option is removed from the document class preamble declaration.
\\
\begin{HlRev}[\ref{memo:example:comment}]
Should you want to emphasise some changes in the manuscript for further check, discussing with the other co-authors, etc., please enclose the text to be highlighted in \ldots
\end{HlRev}
\begin{verbatim}
\begin{HlRev}[\ref{memo:example:comment}]
... something like this. The parameter (here \ref{memo:example:comment}) is optional. It is put after
"Comment" as a side-note. Feel free to pass multiple comma-separated ref's! \end{HlRev}
\end{verbatim}
%
Do you like it? To make it work, make sure to import the \verb|addon/commands-for-revisions.tex| file in your manuscript's main file's preamble, too, and alter the \verb|\externaldocument| command indicating the location of the response letter's file. 
%
\paragraph{References.}
Please notice that to refer to comments within the document, they first must be labelled with an optional parameter:
\begin{verbatim}
\begin{ReviewerComment}[\label{memo:example:comment}]
\end{verbatim}
Later on, also the \verb|\cref{memo:example:comment}| command can be used, to automatically obtain: \cref{memo:example:comment}.

To mention sections, figures, etc., please use the \verb|\ref| command as usual: When referring to the manuscript, prepend \verb|paper:| as a prefix before the actual section/figure/etc. label ID. For instance, the ``\ref{paper:sec:introduction}'' of ``Section~\ref{paper:sec:introduction}''%
\footnote{If you read ``??'' it must be because the referenced paper has no section labelled as ``sec:introduction''.
}
is here generated by this {\LaTeX} code:
\verb|\ref{paper:sec:introduction}|
%
Please alter the line in \texttt{letter.tex} according to the location of the main paper file, so as to let the reference be found:
\begin{verbatim}
	\externaldocument[paper:]{../manuscript}
\end{verbatim}
Bibliographic references are, of course, welcome (e.g., we can mention here a few articles and conference papers whose revisions were accompanied by letters created with this template~\cite{Yeshchenko.etal/TVCG2022:VisualDriftDetectionEventSequenceDataBusinessProcesses,Cecconi.etal/IS2024:MeasuringRuleBasedLTLfSpecifications,Basile.DiCiccio/BPM2025:SecrecyPreservationOnline,}). See the lines at the end of the \textit{letter.tex} (main) file.
%
\paragraph{Other ``comment'' environments.}
Notice that there are several environments other than \verb|ReviewerComment| to specify the role of who wrote the review, such as \verb|GuestEditorComment|, \verb|MetaReviewComment|, \verb|EditorComment|, and the others you can find in: \texttt{addon/commands-for-letter-to-reviewers.tex}. It is advisable not to introduce those with \verb*|\ReviewerSection| unless you want the reviewer's counter to increase accordingly. A regular \verb*|\section*{...}| (e.g., \verb*|\section*{Meta-review}| or \verb|\section*{Editor's comments}|) will do. Comment headers will then be printed without any preceding roman digits (e.g., numbered with \arabic{commentcnt} in place of \Roman{thereviewer}.\arabic{commentcnt}).
\end{sloppypar}
In the following, a couple examples are given. Later on in this page, the different levels of difficulty and the statuses of the referred modifications are listed. Look at the {\LaTeX} code of this page for more information.
\end{Answer}
%
%%%%%%%%%%%%%%%%%%%%%%%%%%%%%%%%%%%%%%%%%%%%%%%%%%%%%%%%%%%%%%%%%%%%%%%%%%%%%%%%%%%%%%%%%%%%%%%%%%%%%%%%%%%%%%%%%%%%%%%%%%%%%%%%
%
\begin{ReviewerComment}
	Another comment of Reviewer \Roman{thereviewer}.
\end{ReviewerComment}
%
\begin{Answer}
	Reply to Reviewer \Roman{thereviewer}.
	\Assignment{Author} \NotEstimatedRevTask
\end{Answer}
%
%%%%%%%%%%%%%%%%%%%%%%%%%%%%%%%%%%%%%%%%%%%%%%%%%%%%%%%%%%%%%%%%%%%%%%%%%%%%%%%%%%%%%%%%%%%%%%%%%%%%%%%%%%%%%%%%%%%%%%%%%%%%%%%%
%
\begin{ReviewerComment}
Start of the reviewer's comment.
\end{ReviewerComment}
%
\begin{AnswerInBetween}
In-between reply of the authors.
\Assignment{Author} \HardRevTask
\end{AnswerInBetween}
%
%%%%%%%%%%%%%%%%%%%%%%%%%%%%%%%%%%%%%%%%%%%%%%%%%%%%%%%%%%%%%%%%
%
\begin{ReviewerCommentReprise}
Reprise of the reviewer's comment.
\end{ReviewerCommentReprise}
%
\begin{AnswerInBetween}
In-between (or final) reply of the authors.
\Assignment{Author} \RevTaskDone
\end{AnswerInBetween}
\Assignment{Author}
\\
\NotEstimatedRevTask \EasyRevTask \MediumRevTask \TimeConsumingRevTask \HardRevTask \DeathRevTask 
\\
\WorkInProgressRevTask \AlmostDoneRevTask \RevTaskDone
